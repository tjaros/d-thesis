\documentclass{mimosis}

\usepackage{metalogo}
\usepackage{xargs}                      % Use more than one optional parameter in a new commands
\usepackage{indentfirst}
%
\usepackage[colorinlistoftodos,prependcaption,textsize=tiny]{todonotes}
\newcommandx{\unsure}[2][1=]{\todo[linecolor=red,backgroundcolor=red!25,bordercolor=red,#1]{#2}}
\newcommandx{\change}[2][1=]{\todo[linecolor=blue,backgroundcolor=blue!25,bordercolor=blue,#1]{#2}}

\newcommandx{\info}[2][1=]{\todo[linecolor=OliveGreen,backgroundcolor=OliveGreen!25,bordercolor=OliveGreen,#1]{#2}}
\newcommandx{\improvement}[2][1=]{\todo[linecolor=Plum,backgroundcolor=Plum!25,bordercolor=Plum,#1]{#2}}
\newcommandx{\thiswillnotshow}[2][1=]{\todo[disable,#2]{#1}}
%

\usepackage{float}
\usepackage{geometry}
\usepackage{babel}
\graphicspath{ {./img/} }

\newcommand{\alphabet}{%
  abcdefghijklmnopqrstuvwxyz%
}
\newlength{\textW}
\setlength{\textW}{\widthof{\alphabet}* \real{2.5}}
\geometry{textwidth=\textW,}
\newcommand{\myref}[2]{\hyperref[#2]{#1~\ref*{#2}}}
%%%%%%%%%%%%%%%%%%%%%%%%%%%%%%%%%%%%%%%%%%%%%%%%%%%%%%%%%%%%%%%%%%%%%%%%
% Some of my favourite personal adjustments
%%%%%%%%%%%%%%%%%%%%%%%%%%%%%%%%%%%%%%%%%%%%%%%%%%%%%%%%%%%%%%%%%%%%%%%%
%
% These are the adjustments that I consider necessary for typesetting
% a nice thesis. However, they are *not* included in the template, as
% I do not want to force you to use them.

% This ensures that I am able to typeset bold font in table while still aligning the numbers
% correctly.
\usepackage{etoolbox}
\usepackage{listings}
\lstset{
  basicstyle=\ttfamily,
  columns=fullflexible,
  frame=single,
  keywordstyle=\color{red},
  breaklines=true,
  postbreak=\mbox{\textcolor{red}{$\hookrightarrow$}\space},
}


\usepackage{dirtree}

\usepackage[binary-units=true]{siunitx}
\DeclareSIUnit\px{px}

\sisetup{%
  detect-all           = true,
  detect-family        = true,
  detect-mode          = true,
  detect-shape         = true,
  detect-weight        = true,
  detect-inline-weight = math,
}

%%%%%%%%%%%%%%%%%%%%%%%%%%%%%%%%%%%%%%%%%%%%%%%%%%%%%%%%%%%%%%%%%%%%%%%%
% Hyperlinks & bookmarks
%%%%%%%%%%%%%%%%%%%%%%%%%%%%%%%%%%%%%%%%%%%%%%%%%%%%%%%%%%%%%%%%%%%%%%%%

\usepackage[%
  colorlinks = true,
  citecolor  = OliveGreen,
  linkcolor  = Mahogany,
  urlcolor   = RoyalBlue,
  unicode,
  ]{hyperref}


\newcommand{\smallquote}[1]{
    \begin{center}
        \begin{minipage}{0.5\textwidth}
            \begin{small}
                #1
            \end{small}
        \end{minipage}
        \vspace{0.5cm}
    \end{center}
}

\usepackage{bookmark}

\usepackage[all]{nowidow}

%%%%%%%%%%%%%%%%%%%%%%%%%%%%%%%%%%%%%%%%%%%%%%%%%%%%%%%%%%%%%%%%%%%%%%%%
% Bibliography
%%%%%%%%%%%%%%%%%%%%%%%%%%%%%%%%%%%%%%%%%%%%%%%%%%%%%%%%%%%%%%%%%%%%%%%%
%
% I like the bibliography to be extremely plain, showing only a numeric
% identifier and citing everything in simple brackets. The first names,
% if present, will be initialized. DOIs and URLs will be preserved.

\usepackage[%
  autocite     = plain,
  backend      = biber,
  doi          = true,
  url          = true,
  giveninits   = true,
  hyperref     = true,
  maxbibnames  = 99,
  maxcitenames = 99,
  sortcites    = true,
  style        = iso-numeric,
  ]{biblatex}

\input{bibliography-mimosis}
\addbibresource{Thesis.bib}

%%%%%%%%%%%%%%%%%%%%%%%%%%%%%%%%%%%%%%%%%%%%%%%%%%%%%%%%%%%%%%%%%%%%%%%%
% Fonts
%%%%%%%%%%%%%%%%%%%%%%%%%%%%%%%%%%%%%%%%%%%%%%%%%%%%%%%%%%%%%%%%%%%%%%%%

\ifxetexorluatex
  \setmainfont{Minion Pro}
\else
  \usepackage[lf]{ebgaramond}
  \usepackage[oldstyle,scale=0.7]{sourcecodepro}
  \singlespacing
\fi

\renewcommand{\th}{\textsuperscript{\textup{th}}\xspace}


\makeindex
\makeglossaries

%%%%%%%%%%%%%%%%%%%%%%%%%%%%%%%%%%%%%%%%%%%%%%%%%%%%%%%%%%%%%%%%%%%%%%%%
% Incipit
%%%%%%%%%%%%%%%%%%%%%%%%%%%%%%%%%%%%%%%%%%%%%%%%%%%%%%%%%%%%%%%%%%%%%%%%

\title{Trusted Platform Modules: visualization of the performance data}
\author{Tomáš Jaroš}

% This ensures that the subsequent sections are being included as root
% items in the bookmark structure of your PDF reader.
\begin{document}
\frontmatter
    \begin{titlepage}
  \vspace*{2cm}
  \makeatletter
  \begin{center}
      \begin{LARGE}
          \textsc{Masaryk University}
      \end{LARGE}\\[0.01cm]
      \begin{Large}
          \textsc{Faculty of Informatics\\}
      \end{Large}
      \vspace*{1cm}
      \includegraphics[width=40mm]{fithesis-fi.pdf}\\
      \vspace*{3cm}
    \begin{Huge}
      \@title
    \end{Huge}\\[1.5cm]
    %
    \begin{Large}
        \textsc{Diploma Thesis}
    \end{Large}\\[1.5cm]
    %
    \begin{LARGE}
        \@author
    \end{LARGE}
    %
    \vfill
  \end{center}
  \begin{flushright}
      \begin{large}
        May 2022
      \end{large}
  \end{flushright}
  \makeatother
\end{titlepage}

\newpage
\null
\thispagestyle{empty}
\newpage


    %\chapter*{Declaration}

\noindent
Hereby I declare, that this paper is my original authorial work, which I
have worked out on my own. All sources, references and literature used or
excerpted during elaboration of this work are properly cited and listed in
complete reference to the due source.

\vspace{1cm}
\begin{flushright}
    Tomáš Jaroš
\end{flushright}
\vfill
\textbf{Advisor:} doc. RNDr. Petr Švenda, Ph.D.

    %\chapter*{Acknowledgements}
\noindent
I want to express my utmost gratitude to my advisor, doc. Petr Švenda, for his guidance, patience, and helpful advice he provided me during the preparation of this thesis. I would also like to thank my family for their support and encouragement throughout my studies.
    %\chapter*{Abstract}
\noindent
%Táto bakalárska práca prezentuje prehľad technológie Trusted Platform Module. Postupne objasňuje vybrané koncepty súvisiace s touto technológiou, poskytuje prehľad o jej možných využitiach a zasadzuje túto prácu do kontextu výkonnostnej analýzy TPM čipov a kryptografických čipových kariet. Praktickým výstupom tejto práce je návrh a implementácia nástroja slúžiaceho na vizualizáciu dát zozbieraných nástrojmi z projektov tpm2-algtest a JCAlgTest slúžiacich na výkonnostné testovanie TPM 2.0 čipov a čipových kariet s platformou JavaCard.
This thesis presents an overview of selected concepts related to Trusted Platform Module technology, outlines its potential use cases, and places this work in the context of performance analysis of TPM chips and cryptographic smart cards. The contribution of this thesis is a tool for visualization of data collected by tools from the tpm2-algtest and JCAlgTest projects used for performance testing of TPM 2.0 chips and JavaCard smart cards.

\section*{Keywords}
\noindent
Trusted Platform Module, TPM, smart card, visualization, tpm2-algtest, JCAlgTest

    \tableofcontents
    \listoffigures
    \glsaddall
    %\printglossary[type=\acronymtype]
    \chapter{Introduction}
\todo{Introduction and Conclusion are written as last anyway so I still have a lot of time}

\todo{Describe Fault injection and side channel analysis with regards to hardware security or in general}

\todo{Describe place of post-quantum crypto in modern world, dangers of classical cryptography in presence of quantum computers, standardization, ....}


\todo{Describe the objective of the thesis ? Well you can do it when you actually know what you gonna do, regarding fault injection}

\todo{Provide some quick sketch of how the thesis content is divided.}


\mainmatter
  \chapter{State of the Art}

\section{Trusted Platform Module}
The Trusted Platform Module is a system component used as a cryptographic co-processor. It was developed by and standardized by the Trusted Computing Group (TCG) consortium with the purpose of laying a foundation on which secure systems could be further created and developed. 

\subsection{History}
The first broadly used specification was TPM 1.1b, released in 2003. TPMs released under this specification already provided some essential functions found in modern TPMs consisting of key generation of RSA key pairs, storage, secure authorisation, and device-health attestation. For the sake of assuring privacy, the
use of anonymous identity keys based on certificates was introduced. In order to take advantage of such
functionality, these certificates needed to be provided with the TPM, and any generation of such keys was available only after owner authorisation. To be able to anonymously prove the origin of the keys generated
by TPM, a \texttt{privacy certification authority} was created. The integrity of measurements collected during systems boot sequence is provided by Platform Configuration Registers (PCRs). Both PCRs and identity keys might have been used to prove the health of the system's boot sequence~\cite{arthur2015practical}.


The hardware specification was not standardized in TPM 1.1b. This caused various incompatibilities. The TPMs across different vendors provided differing interfaces, which required different drivers. Pin-outs on the chips were not prescribed by any standard. Additionally, there were no countermeasures against dictionary attacks~\cite{arthur2015practical}.
% Should i also mention DAA ?

While being in development from 2005 to 2009, the TPM 1.2 specification went through numerous releases. Regarding the need to store shipped certificates for TPM's endorsement keys on a hard disk, about 2KB of non-volatile RAM was added. A new design needed to be made to support key migration between different TPMs because the old design of key migration would in TPM 1.12 require users to have TPM owner authorization. The new idea made users able to create migratable keys and then relied on a designated third party that could exclusively migrate such keys. Said keys could also be certified. Thus, they were called Certified Migratable keys. Additionally, an internal timer able to synchronize with the external one was added in 1.12, which has its use when signing data due to timestamps. Version 1.12 required API to provide backward compatibility for 1.1b. This increased the complexity of the new specification. The TPM 1.12 became widely used in x86 personal computers starting 2005 and later in 2008 also in servers~\cite{arthur2015practical}.

One of the factors that contributed to the need for yet another specification after TPM 1.12 was that in 2005, some substantial collision attacks were found against the SHA-1 hash function. Analysis regarding the use of SHA-1 in TPM revealed the attacks not being applicable~\cite{tcg_tpm1.12_sha-1_uses}. Due to the extensive use of SHA-1 in TPM 1.12, the new specification had to permit any hashing algorithm without the need to make any changes to the specification. A so-called \texttt{digest agility}. Another problem was the lack of a symmetric algorithm required in the TPM specification. The use of RSA for encryption of serialized data was impractical because RSA operations are slow. Neither would help support bigger-sized RSA keys because that would cause a higher chip cost, incompatibility issues, and lower performance. That's why it was decided that the following specification would adopt support for symmetric encryption, which is faster and more suitable for encryption of large byte streams. Having this many problems, an overhaul of the specification would be convenient. And the architects of TPM 2.0 took advantage of the situation~\cite{arthur2015practical}.


\subsection{Features}


\section{Performance analysis}
\subsection{tpm2-algtest}
The \texttt{tpm2-algtest}\footnote{https://github.com/crocs-muni/tpm2-algtest} is a tool for automatic gathering of information about the TPM2 devices \cite{Struk2019thesis}. The tool uses libraries implementing Trusted Computing Group's TPM2 Software Stack\footnote{https://github.com/tpm2-software/tpm2-tss} which allows for simplification of development when programming applications supposed to interact with the TPM. The tool tests for support of specific commands and supporting routines, values of structures defined in the TPM 2.0 specification \cite{tcg_p3_commands, tcg_p4_supproutines, tcg_p2_structures}. Supported cryptographic algorithms are also subject to performance analysis where the time to execute such an algorithm is repeatedly measured and recorded. Additionally, the tool uses the TPM to generate key pairs for RSA and ECC-based algorithms so that they can be further analyzed by various means.
  
  \chapter{Analysis and design}
One of the objectives of this work was the creation of a tool for processing and visualization of results from TPM devices. Such a tool already exists, but it is hard to extend it, and it needs extensive refactoring, so it would be more beneficial to redesign it.

This chapter describes the original tool for generating visualizations and discusses its problems and possible improvements. Then I discuss the design of the new application to replace the existing one.

\section{JCAlgTest}
The \texttt{JCAlgTest}\footnote{https://github.com/crocs-muni/JCAlgTest} is a suite of tools used for automated analysis of cryptographic smartcards, specifically those running the JavaCard platform. It takes advantage of the officially available JavaCard API to test for support of specific algorithms, measures their performance characteristics, and collects general information about the tested smart card. The suite consists of three modules:
\begin{itemize}
  \item
        \texttt{JCAlgTestJavaCard} A JavaCard applet that needs to be uploaded onto the smart card we want to test. According to instructions from \texttt{JCAlgTestClient} application to execute code for specified operation. Firstly it tries to instantiate a particular object belonging to the operation. If it succeeds, that means that the operation is supported, and in the case of performance testing, it can proceed further with the execution. If the instantiation was unsuccessful, a specific exception is thrown, which usually means that the operation is not supported.
  \item
        \texttt{JCAlgTestClient} A host application which is responsible for communication with \texttt{JCAlgTestJavaCard} applet via APDU commands and responses. The information is gathered and recorded. The execution time measurement during the performance testing  is estimated externally due to the limitations of the JavaCard platform, which does not support any time measurement methods.
  \item
        \texttt{JCAlgTestProcess} Application which is used for visualization of JavaCard and TPM results. Results in the form of CSV files are processed into various kinds of visualizations and tabular data. As the state of the tool is deemed obsolete, its reimplementation was needed and was performed as one of the goals of this thesis. The reasons behind the reimplementation and the new design will be discussed in the following sections.
\end{itemize}

\section{JCAlgTest process}
The \texttt{JCAlgTest process} is an application used for tabular and chart visualisations of JavaCard smart card and TPM 2.0 run time data created by \texttt{JCAlgTest} and \texttt{tpm2-algtest} tools respectively. 

This section will describe the application's current state and its problems.
\subsection{Current state}
The application is written in Java programming language. It is designed mainly as a command-line tool for generating HTML visualizations for JavaCard smart card run time data. Still, it also contains means of visualizing TPM 2.0 run time data. The project's website\footnote{\url{http://jcalgtest.org}} shows JavaCart smart card visualisations. Each type of visualization is performed by building an HTML file and either by embedding JavaScript calls to visualization functions contained in  \texttt{D3.js library} or \texttt{Google Charts API} or by creating a basic HTML table with added CSS styling and quality of life improvements such as sorting or filtering of entries using JavaScript. An \texttt{Apache Ant} command-line tool is used to build the application. The command line API is straightforward. The user needs to provide input and output directories and the type of visualization he wants to generate. 

\subsection{Problems}
Whole \texttt{JCAlgTest} suite was developed over several years by various people. Initial version of \texttt{JCAlgTest process} was created by Petr Švenda, my supervisor, along with rest of \texttt{JCAlgTest} suite. Then Rudolf Kvašňovský further developed the \texttt{JCAlgTest process} tool as a part of his bachelors thesis. From there on several smaller updates from various people were performed. The app serves its intended purpose, however there are some problems which make hard to expand it.

One of the problems of the application is that there is no object representation of device profiles. The data is mostly stored in simple arrays. That is the reason why a relatively simple task of extending functionality may prove very difficult. Another problem is the amount of hard coded HTML tags into strings. One unintentional mistake of deleted angle bracket may corrupt whole HTML output.

\section{Improvements}

  \chapter{AlgTest process tool}
In this chapter, we describe the development details of the tool called \texttt{AlgTest Process} of which re-implementation was performed. The tasks for which the tool is responsible are processing the datasets created by automatic testing tools targeted at JavaCard smart cards and Trusted Platform Modules and generating various outputs from these datasets in HTML files.

\begin{figure}[h]
    \centering
    \includegraphics[width=\textwidth]{img/scheme.png}
    \caption{Scheme of AlgTest process application}
    \label{fig:algtest-process-scheme}
\end{figure}


\section{Parsing}
The data, namely the run time results we are working with, were collected over the years in parallel with the development of the tools which generate them. Consequentially the data does not have the same form, as there is no prescribed schema. Therefore it is convenient to design suitable means of accessing it. In the re-implementation of the app, the mapping of CSV files into corresponding classes was introduced. This approach makes it significantly easier to manipulate the data, and the resulting code readability is better.

\section{Results}
The tool produces various types of visualization and tables in HTML pages. These results simplify getting information about the tested devices for the human reader. The contents are made responsive, and some visualizations are more interactive.

This section describes each output of the tool.

\subsection{Algorithm execution time}
Algorithm execution time pages are generated for each card separately and present an alternative for searching in the corresponding raw CSV file. The page contains metadata detailing the test performed on this smart card, followed by tables containing algorithm measurement values. There are multiple tables because the tested algorithms are divided into numerous categories. A quick link for each of these categories is provided as a utility to aid searching on the page.

\begin{figure}[h]
    \centering
    \includegraphics[width=\textwidth]{img/NXP JCOP4 P71D321 execution-time table.png}
    \caption{Algorithm execution timetable of NXP JCOP4 P71D321}
    \label{fig:execution-time-table}
\end{figure}

\subsection{Comparative table}
The comparative table allows for a simple comparison between cards. Multiple comparative tables are generated in one file divided into symmetric, asymmetric cryptography, and frequently used algorithms, also called Top Functions. The tables themselves allow for sorting of the cards just by simply clicking on the column containing values according to which rows will be sorted.

\begin{figure}[h]
    \centering
    \includegraphics[width=\textwidth]{img/comparative-table.png}
    \caption{Comparative table}
    \label{fig:comparative-table}
\end{figure}

\subsection{Scalability graphs}
Scalability graphs illustrate a single card's performance for a specific algorithm depending on input data length. The graph is generated for each method that was measured. The time scale from minimum to maximum measurement value for each measurement of the given algorithm and average time both in milliseconds can be found by simply hovering over the point in the graph.

\begin{figure}[H]
    \centering  
    \includegraphics[width=\textwidth-3.5cm]{img/NXP_JCOP_J3D081_v2.4.2R2 scalability graph.png}
    \caption{
    Scalability chart of NXP JCOP J3D081 v2.4.2R2 showing performance of DES algorithm in CBC mode, for data lengths We can observe 16, 32, 64, 128 bits the performance seems to be constant, following 256, 512 bits an almost linear increase.
    }
    \label{fig:scalability-chart}
\end{figure}

\subsection{Radar graphs}
Radar graphs visualize the card's performance compared to all other tested cards. A percentage describes overall performance on a scale of 0 to 100, where closer to 100 is faster. Zero value signifies either that the algorithm is not supported or not successfully tested. By hovering over the points on edge, we can see the actual average operation time for the chosen algorithm. The algorithms chosen for comparison were selected as most frequently used, and in the context of the tool, we call them Top Functions. 

\begin{figure}[h]
    \centering
    \includegraphics[width=\textwidth]{img/JC30M48CR radar graph.png}
    \caption{
    The radar graph of the JC30M48CR card  can be considered excellent, apart from no evident support for algorithms involving Elliptic Curve Cryptography signified by a cut-out at the bottom of the graph.
    }
    \label{fig:radar-graph}
\end{figure}

\subsection{Support table}
The support table contains information about support for a particular algorithm from the specific smart card. During the test for algorithm support, we can get multiple results. Either the algorithm is supported,  unsupported, error was returned, or lastly, the algorithm was not tested. Each of these results is denoted by the content and the color of the cell. By hovering over the error cell, we can find out what kind of exception or error was returned from the card. The table also provides means to filter the cards according to their support for categories of algorithms.

\begin{figure}[h]
    \centering
    \includegraphics[width=\textwidth]{img/support-table.png}
    \caption{Support table}
    \label{fig:support-table}
\end{figure}

\subsection{Similarity table}
The similarity table shows how a chosen pair of cards differ in the performance of a selected group of operations. The similarity is computed for six different categories of algorithms and results in a percentage on a scale of 0 to 100, also signified by the color of each cell. The higher number indicates higher performance similarity. The cards in the table are sorted so that most similar cards are in the top left corner and the least similar are in the bottom right corner.

\begin{figure}[h]
    \centering
    \includegraphics[width=\textwidth]{img/similarity-table.png}
    \caption{Similarity table}
    \label{fig:similarity-table}
\end{figure}

\section{Tools}
The \texttt{AlgTest process} tool incorporates various instruments to produce desired outputs. As the tool itself is written in Python it is able to utilize various packages from \texttt{PyPI}\footnote{\url{https://pypi.org/}} to aid future extensions.

\begin{itemize}
    \item For creation and manipulation of HTML documents a Python library called \texttt{dominate}\footnote{\url{https://pypi.org/project/dominate/}} is used. Its usage is relatively straightforward, and the code you write is in pure python instead of using additional templating language as in popular web frameworks.
    \item The styling is taken care of by a CSS framework called \texttt{bootstrap}\footnote{\url{https://getbootstrap.com/}} which allows for designing responsive web pages using premade CSS classes and Javascript. The framework itself is extensively used and regularly updated with new versions.
    \item \texttt{Google Charts API}\footnote{\url{https://developers.google.com/chart}}  is utilized for Scalability graph visualisation.
    \item \texttt{D3.js library}\footnote{\url{https://d3js.org/}}  is used for visualisation of Radar graphs.
\end{itemize}


  
  \bookmarksetup{startatroot}

  \chapter{Conclusion}
% Conclusion and future work.
\todo{Well come back when u done with the thesis, next year :DDD}

  %% \appendix
\renewcommand{\thechapter}{A}
\chapter{Usage of AlgTest pyProcess}

\definecolor{codegreen}{rgb}{0,0.6,0}
\definecolor{codegray}{rgb}{0.5,0.5,0.5}
\definecolor{codepurple}{rgb}{0.58,0,0.82}
\definecolor{backcolour}{rgb}{0.95,0.95,0.92}

\lstdefinestyle{mystyle}{
    backgroundcolor=\color{white},   
    commentstyle=\color{codegreen},
    keywordstyle=\color{magenta},
    numberstyle=\tiny\color{codegray},
    stringstyle=\color{codepurple},
    basicstyle=\ttfamily\footnotesize,
    breakatwhitespace=false,         
    breaklines=true,                 
    captionpos=b,                    
    keepspaces=true,                          
    showspaces=false,                
    showstringspaces=false,
    showtabs=false,                  
    tabsize=2
}

\lstset{style=mystyle}

\section{Installation}
After cloning into the repository containing \texttt{AlgTest process} tool, it is necessary to install a few dependencies. The project includes a script called \texttt{setup.py} which, when ran installs required packages. It may be convenient to create a Python virtual environment that allows us to install and manage Python project packages locally without unnecessary global installation. It is important to create and source a virtual environment before running the setup script.
\begin{lstlisting}[language=bash]
    $ python -m venv venv
    $ source venv/bin/activate
    $ python setup.py
\end{lstlisting}

\section{Folder structure}
\begin{itemize}
    \item \texttt{algtestprocess}
        \begin{itemize}
            \item \texttt{modules} -- folder containing \texttt{algtestprocess} modules.
                \begin{itemize}
                    \item \texttt{components} -- folder containing reusable code for parts of HTML documents. Each file contains classes, constants related to one component.
                    \item \texttt{pages} -- folder containing files with classes corresponding to generated HTML pages, each page has a separate file containing classes that are responsible for page creation.
                    \item \texttt{visualization} -- folder containing specific standalone visualizations which can be used independently or as a part of created pages.
                    \item \texttt{parser}
                        \begin{itemize}
                            \item \texttt{javacard} -- folder containing parser implementations for JavaCard profiles
                            \item \texttt{tpm} -- folder containing parser implementations for TPM profiles
                        \end{itemize}
                    \item \texttt{config.py} -- file containing classes for choice of algorithms used in specific pages
                    \item \texttt{jcalgtest.py} -- file containing classes for storage of Java Card device profiles
                    \item \texttt{tpmalgtest.py} -- file containing classes for storage of TPM device profiles
                \end{itemize}
        \end{itemize}
    \item \texttt{assets} -- folder which contains CSS and JavaScript assets that are used in visualizations, the folder was created from CSS and JavaScript assets which were used by \texttt{JCAlgTest process}.
    \item \texttt{setup.py} -- script for installation of dependencies and download of results from their official GitHub repository\footnote{\url{https://github.com/crocs-muni/jcalgtest_results}}, which is an official repository for JavaCard datasets, but also contains some TPM datasets. This repository at the time of writing contains only a subset of tested TPMs with no cryptographic properties. Complete datasets for TPMs are not publicly available for now. 
    \item \texttt{process.py} -- main entry point for the generation of outputs, its usage is described in the following section
\end{itemize}

\section{Usage}
The script process.py provides following CLI arguments and options:

\begin{itemize}
    \item Device type which is either \texttt{tpm} or \texttt{javacard}.
    \item Set of operations from \texttt{process, all, execution-time, comparative, radar, scalability, similarity, support, compare, and heatmap}. All of them except for the special operations \texttt{process} and \texttt{all} signify a type of visualization.
    
    A set of operations can be selected, but when generating visualizations for JavaCards, then a \texttt{process} operation needs to be run at least once so that some issues with CSV files are fixed, and JSON outputs are generated for the following operations which use them. In the subsequent runs the \texttt{process} argument is not necessary.
    
    The \texttt{all} operation tells the script to generate all possible visualizations for a given device.
    \item \texttt{-i/-{}-results-dir} A directory with device profiles. It is important that profiles of only one device type are contained in the children directories.
    \item \texttt{-o/-{}-output-dir} Output directory to which the HTML outputs are stored. 
\end{itemize}

\begin{lstlisting}[language=bash]
    $ python process.py javacard process support similarity -i ./jcalgtest_results/javacard/ -o ./jcalgtest_results/javacard/web
    $ cp -r ./assets ./jcalgtest_results/javacard/web # Visualizations need assets folder
    $ python process.py tpm all -i ./jcalgtest_results/tpm/ -o ./out
    $ cp -r ./assets ./out
\end{lstlisting}

\renewcommand{\thechapter}{B}
\chapter{Visualizations}\label{appendix:diagrams-visualizations}

\begin{figure}[ht]
  \centering
  \begin{subfigure}{\linewidth}
    \centering
    \includegraphics[width=.8\linewidth]{img/visualizations/tpm-support-properties.png}
    \caption{TPM properties}
  \end{subfigure}
  
  \begin{subfigure}{\linewidth}
    \centering
    \includegraphics[width=.8\linewidth]{img/visualizations/tpm-support-algorithms.png}
    \caption{TPM algorithms}
  \end{subfigure}  

  \begin{subfigure}{\linewidth}
    \centering
    \includegraphics[width=.8\linewidth]{img/visualizations/tpm-support-commands.png}
    \caption{TPM commands}
  \end{subfigure}  
  \caption{Parts of TPM support table}  
\end{figure} 

\begin{landscape}
    \begin{figure}[!t]
        \includegraphics[width=\linewidth, height=\textwidth]{img/visualizations/tpm-execution-time.png}
        \caption{A part of Execution time visualization for INTC Intel 401.1.0.0}
    \end{figure}
\end{landscape}

\begin{landscape}
    \begin{figure}[!t]
        \includegraphics[width=\linewidth, height=\textwidth]{img/visualizations/tpm-similarity.png}
        \caption{A similarity table for TPMs. The table shows only a subset of visualized TPMs.}
    \end{figure}
\end{landscape}

\begin{landscape}
\begin{figure}[!t]
    \centering
    \includegraphics[width=\linewidth]{img/visualizations/INTC_Intel_302.12.0.0 radar graph.png}
    \caption{
    The radar graph of the INTC Intel 302.12.0.0 TPM.
    }
\end{figure}
\end{landscape}

\begin{landscape}
\begin{figure}[!t]
    \centering
    \includegraphics[width=\linewidth]{img/visualizations/INTC_Intel_11.5.0.1058_vs_IFX_SLB9670_7.63.13.6400-radar-comparison.png}
    \caption{
    Comparison radar graph between INTC Intel 11.5.0.1058 and IFX SLB9670 7.63.13.6400.
    }
\end{figure}
\end{landscape}

\begin{figure}[H]
    \centering
    \includegraphics[width=\textwidth,height=\textheight-1.5cm, keepaspectratio]{img/visualizations/rsa.png}
    \caption{TPM heatmap of MSB values for 10000 RSA 1024-bit key pairs}
    \label{fig:heatmap-rsa-10000-1024}
\end{figure}

\begin{figure}[H]
    \centering
    \includegraphics[width=\textwidth,height=\textheight-1.5cm, keepaspectratio]{img/visualizations/rsa1.png}
    \caption{TPM heatmap of MSB values for 100 RSA 2048-bit key pairs}
    \label{fig:heatmap-rsa-100-2048}
\end{figure}

\begin{figure}[H]
    \centering
    \includegraphics[width=\textwidth]{img/visualizations/NXP J3A080-scalability-DES3.png}
    \caption{NXP J30808 JavaCard smart card scalability chart}
\end{figure}

\begin{figure}[H]
    \centering
    \includegraphics[width=\textwidth]{img/visualizations/jc-comparative-table.png}
    \caption{A part of the comparative table for JavaCard smart cards}
    \label{fig:my_label}
\end{figure}


\renewcommand{\thechapter}{C}
\chapter{Diagram and Dataset examples}\label{appendix:diagram-profiles}
\begin{figure}[H]
    \centering
    \includegraphics[width=\textwidth,height=\textheight-6cm, keepaspectratio]{img/diagrams/object_diagram.png}
    \caption{Diagram of Performance and Support device profiles for TPMs and JavaCards.}
    \label{fig:dev-profiles-diagram}
\end{figure}


\begin{figure}
    \centering
    \lstinputlisting[]{img/datasets/tpm-perf/tpm-perf.csv}
    \caption{A part of the performance device profile for TPMs. The initial information in the upper part of the figure corresponds to \texttt{test\_info} attribute of \texttt{ProfileTPM} class, whereas the rest of the figure shows TPM performance results corresponding to \texttt{PerformanceResultTPM} class. Both of the mentioned classes could be seen in \myref{Figure}{fig:dev-profiles-diagram}. Ellipses mean skipped lines and are not a part of any device profiles.}
\end{figure}

\begin{figure}
    \centering
    \lstinputlisting[]{img/datasets/tpm-support/tpm-support.csv}
    \caption{A shortened support device profile for TPMs. Hexadecimal numbers in the lower half of the profile are identifiers of supported algorithms, commands, and curves.}
\end{figure}

\begin{figure}
    \centering
    \lstinputlisting[]{img/datasets/jc-header/jc-tinfo.csv}
    \caption{This is a common part of all device profiles for JavaCard smart cards, initial contents are very similar. All three parts correspond to attributes in \texttt{ProfileJC} class seen in \myref{Figure}{fig:dev-profiles-diagram}. First part corresponds to \texttt{test\_info} attribute, second corresponds to \texttt{jc\_system} attribute and last one corresponds to \texttt{cplc} attribute.}
\end{figure}

\begin{figure}
    \centering
    \lstinputlisting[]{img/datasets/jc-fixed/fix-results.csv}
    \caption{This corresponds to a part of the performance device profile for JavaCard smart cards where fixed data length performance was measured. Noticeably, many items in each measurement correspond to the attributes of \texttt{PerformanceResultJC} class seen in \myref{Figure}{fig:dev-profiles-diagram}.}
\end{figure}

\begin{figure}
    \centering
    \lstinputlisting[]{img/datasets/jc-variable/var-results.csv}
    \caption{This is a part of the performance device profile for JavaCard smart cards, where variable data length performance was measured. As can be seen in the upper half of the figure, some measurements could not be performed because the device does not support such algorithms. In the lower half of the figure, two successful measurements were performed for 16 and 32-byte data lengths.}
\end{figure}

\begin{figure}
    \centering
    \lstinputlisting[]{img/datasets/jc-support/jc-support.csv}
    \caption{A part of the support device profile for JavaCard smart cards.}
\end{figure}

\renewcommand{\thechapter}{D}
\chapter{Data Attachments}
\begin{itemize}
    \item \texttt{AlgTest\_pyProcess} folder containing the implementation of \texttt{AlgTest pyProcess} tool
    \item \texttt{jcalgtest\_visualizations} folder containing the JavaCard smart card visualizations generated by \texttt{AlgTest pyProcess} tool. Stylesheets and JavaScript is provided in \texttt{assets} folder and some decorative images are provided in \texttt{pics} folder. Decorative images were taken from \texttt{crocs-muni/jcalgtest\_results} on GitHub.
    
    \item \texttt{tpmalgtest\_visualizations} folder containing the TPM visualizations generated by \texttt{AlgTest pyProcess} tool. Stylesheets and JavaScript is provided in \texttt{assets} folder 
\end{itemize}

  \backmatter
  \printindex
  \begingroup
  \sloppy

  \printbibliography
\endgroup


\end{document}

