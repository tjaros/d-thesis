\documentclass{mimosis}

\usepackage{metalogo}
\usepackage{xargs}                      % Use more than one optional parameter in a new commands
\usepackage{indentfirst}
%
\usepackage[colorinlistoftodos,prependcaption,textsize=tiny]{todonotes}
\newcommandx{\unsure}[2][1=]{\todo[linecolor=red,backgroundcolor=red!25,bordercolor=red,#1]{#2}}
\newcommandx{\change}[2][1=]{\todo[linecolor=blue,backgroundcolor=blue!25,bordercolor=blue,#1]{#2}}

\newcommandx{\info}[2][1=]{\todo[linecolor=OliveGreen,backgroundcolor=OliveGreen!25,bordercolor=OliveGreen,#1]{#2}}
\newcommandx{\improvement}[2][1=]{\todo[linecolor=Plum,backgroundcolor=Plum!25,bordercolor=Plum,#1]{#2}}
\newcommandx{\thiswillnotshow}[2][1=]{\todo[disable,#2]{#1}}
%

\usepackage{float}
\usepackage{geometry}
\usepackage{babel}
\usepackage{svg}
\graphicspath{ {./img/} }

\newcommand{\alphabet}{%
  abcdefghijklmnopqrstuvwxyz%
}
\newlength{\textW}
\setlength{\textW}{\widthof{\alphabet}* \real{2.5}}
\geometry{textwidth=\textW,}
\newcommand{\myref}[2]{\hyperref[#2]{#1~\ref*{#2}}}
%%%%%%%%%%%%%%%%%%%%%%%%%%%%%%%%%%%%%%%%%%%%%%%%%%%%%%%%%%%%%%%%%%%%%%%%
% Some of my favourite personal adjustments
%%%%%%%%%%%%%%%%%%%%%%%%%%%%%%%%%%%%%%%%%%%%%%%%%%%%%%%%%%%%%%%%%%%%%%%%
%
% These are the adjustments that I consider necessary for typesetting
% a nice thesis. However, they are *not* included in the template, as
% I do not want to force you to use them.

% This ensures that I am able to typeset bold font in table while still aligning the numbers
% correctly.
\usepackage{etoolbox}
\usepackage{listings}
\lstset{
  basicstyle=\ttfamily,
  columns=fullflexible,
  frame=single,
  keywordstyle=\color{red},
  breaklines=true,
  postbreak=\mbox{\textcolor{red}{$\hookrightarrow$}\space},
}


\usepackage{pifont}
\newcommand{\cmark}{\text{\ding{51}}}
\newcommand{\xmark}{\text{\ding{55}}}
\usepackage{adjustbox}


\usepackage[binary-units=true]{siunitx}
\DeclareSIUnit\px{px}

\sisetup{%
  detect-all           = true,
  detect-family        = true,
  detect-mode          = true,
  detect-shape         = true,
  detect-weight        = true,
  detect-inline-weight = math,
}

%%%%%%%%%%%%%%%%%%%%%%%%%%%%%%%%%%%%%%%%%%%%%%%%%%%%%%%%%%%%%%%%%%%%%%%%
% Hyperlinks & bookmarks
%%%%%%%%%%%%%%%%%%%%%%%%%%%%%%%%%%%%%%%%%%%%%%%%%%%%%%%%%%%%%%%%%%%%%%%%

\usepackage[%
  colorlinks = true,
  citecolor  = OliveGreen,
  linkcolor  = Mahogany,
  urlcolor   = RoyalBlue,
  unicode,
  ]{hyperref}


\newcommand{\smallquote}[1]{
    \begin{center}
        \begin{minipage}{0.5\textwidth}
            \begin{small}
                #1
            \end{small}
        \end{minipage}
        \vspace{0.5cm}
    \end{center}
}

\usepackage{bookmark}

\usepackage[all]{nowidow}

\usepackage{pdflscape}
\usepackage{subcaption}
\usepackage{color, colortbl}
\usepackage{amssymb}

\definecolor{Gray}{gray}{0.9}

\widowpenalty = 10000
\clubpenalty = 10000
\postdisplaypenalty = 100

\parskip0pt plus 6pt
%%%%%%%%%%%%%%%%%%%%%%%%%%%%%%%%%%%%%%%%%%%%%%%%%%%%%%%%%%%%%%%%%%%%%%%%
% Bibliography
%%%%%%%%%%%%%%%%%%%%%%%%%%%%%%%%%%%%%%%%%%%%%%%%%%%%%%%%%%%%%%%%%%%%%%%%
%
% I like the bibliography to be extremely plain, showing only a numeric
% identifier and citing everything in simple brackets. The first names,
% if present, will be initialized. DOIs and URLs will be preserved.

\usepackage[%
  autocite     = plain,
  backend      = biber,
  doi          = true,
  url          = true,
  giveninits   = true,
  hyperref     = true,
  maxbibnames  = 99,
  maxcitenames = 99,
  sortcites    = true,
  style        = iso-numeric,
  ]{biblatex}

\input{bibliography-mimosis}
\addbibresource{Thesis.bib}

%%%%%%%%%%%%%%%%%%%%%%%%%%%%%%%%%%%%%%%%%%%%%%%%%%%%%%%%%%%%%%%%%%%%%%%%
% Fonts
%%%%%%%%%%%%%%%%%%%%%%%%%%%%%%%%%%%%%%%%%%%%%%%%%%%%%%%%%%%%%%%%%%%%%%%%

\ifxetexorluatex
  \setmainfont{Minion Pro}
\else
  \usepackage[lf]{ebgaramond}
  \usepackage[oldstyle,scale=0.7]{sourcecodepro}
  \singlespacing
\fi

\renewcommand{\th}{\textsuperscript{\textup{th}}\xspace}


\makeindex
%\makeglossaries

%%%%%%%%%%%%%%%%%%%%%%%%%%%%%%%%%%%%%%%%%%%%%%%%%%%%%%%%%%%%%%%%%%%%%%%%
% Incipit
%%%%%%%%%%%%%%%%%%%%%%%%%%%%%%%%%%%%%%%%%%%%%%%%%%%%%%%%%%%%%%%%%%%%%%%%

\title{Investigating and extending disorientation faults attacks against CSIDH}
\author{Tomáš Jaroš}

% This ensures that the subsequent sections are being included as root
% items in the bookmark structure of your PDF reader.
\begin{document}
\setcounter{tocdepth}{1}
\frontmatter
    \begin{titlepage}
  \vspace*{2cm}
  \makeatletter
  \begin{center}
      \begin{LARGE}
          \textsc{Masaryk University}
      \end{LARGE}\\[0.01cm]
      \begin{Large}
          \textsc{Faculty of Informatics\\}
      \end{Large}
      \vspace*{1cm}
      \includegraphics[width=40mm]{fithesis-fi.pdf}\\
      \vspace*{3cm}
    \begin{Huge}
      \@title
    \end{Huge}\\[1.5cm]
    %
    \begin{Large}
        \textsc{Diploma Thesis}
    \end{Large}\\[1.5cm]
    %
    \begin{LARGE}
        \@author
    \end{LARGE}
    %
    \vfill
  \end{center}
  \begin{flushright}
      \begin{large}
        May 2022
      \end{large}
  \end{flushright}
  \makeatother
\end{titlepage}

\newpage
\null
\thispagestyle{empty}
\newpage


    \chapter*{Declaration}

\noindent
Hereby I declare, that this thesis is my original work, which I
have worked out on my own. All sources, references and literature used or
excerpted during elaboration of this work are properly cited and listed in
complete reference to the due source.

\vspace{1cm}
\begin{flushright}
    Tomáš Jaroš
\end{flushright}
\vfill
\textbf{Advisor:} Lukasz Michal Chmielewski, PhD

    \chapter*{Acknowledgements}
\noindent
\todo{Write acknowledgements to Lukasz, Krijn, ... and mom and dad ofc}
%I want to express my utmost gratitude to my advisor, Dr. Lukasz Chmielewski, and consultant  Krijn Reinders, for their professional guidance, patience, and helpful advice he provided me during the preparation of this thesis. I would also like to thank my family for their immense support and encouragement throughout my studies.

    \chapter*{Abstract}
\noindent
\todo{Change this when you already know what the thesis will be actually about lol}
\section*{Keywords}
\noindent
Post-quantum cryptography, CSIDH, Isogeny cryptography, Fault injection, Disorientation faults

    \tableofcontents
    \listoffigures
    \chapter{State of the Art}
\todo{Here we may want to mention, the state of the PQC schemes in general, research with regards to Side-Channel and Fault injection analysis in general and on PQC schemes, not sure yet.}

    \chapter{CSIDH}
\todo{Here we introduce the CSIDH scheme. As I am not mathematician, I can probably get by with skipping some harder concepts and instead refer to the original publication. Then I can mention side channel research on CSIDH, disorientation faults, etc.}

    %\glsaddall
    %\printglossary[type=\acronymtype]
    \chapter{Introduction}
\todo{Introduction and Conclusion are written as last anyway so I still have a lot of time}

\todo{Describe Fault injection and side channel analysis with regards to hardware security or in general}

\todo{Describe place of post-quantum crypto in modern world, dangers of classical cryptography in presence of quantum computers, standardization, ....}


\todo{Describe the objective of the thesis ? Well you can do it when you actually know what you gonna do, regarding fault injection}

\todo{Provide some quick sketch of how the thesis content is divided.}


\mainmatter
  
  \bookmarksetup{startatroot}

  \chapter{Conclusion}
% Conclusion and future work.
\todo{Well come back when u done with the thesis, next year :DDD}

  % \appendix
\renewcommand{\thechapter}{A}
\chapter{?}\label{appendix:?}

\definecolor{codegreen}{rgb}{0,0.6,0}
\definecolor{codegray}{rgb}{0.5,0.5,0.5}
\definecolor{codepurple}{rgb}{0.58,0,0.82}
\definecolor{backcolour}{rgb}{0.95,0.95,0.92}

\lstdefinestyle{mystyle}{
    backgroundcolor=\color{white},   
    commentstyle=\color{codegreen},
    keywordstyle=\color{magenta},
    numberstyle=\tiny\color{codegray},
    stringstyle=\color{codepurple},
    basicstyle=\ttfamily\footnotesize,
    breakatwhitespace=false,         
    breaklines=true,                 
    captionpos=b,                    
    keepspaces=true,                          
    showspaces=false,                
    showstringspaces=false,
    showtabs=false,                  
    tabsize=2
}

\lstset{style=mystyle}

\renewcommand{\thechapter}{B}
\chapter{?}\label{appendix:?}

\renewcommand{\thechapter}{C}
\chapter{?}\label{appendix:?}

  \backmatter
  \printindex
  \begingroup
  \sloppy

  \printbibliography
\endgroup


\end{document}

